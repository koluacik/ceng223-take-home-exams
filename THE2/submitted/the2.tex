\documentclass[11pt]{article}
\usepackage[utf8]{inputenc}
\usepackage{float}
\usepackage{amsmath}
\usepackage{enumitem}
\usepackage{amssymb}
\usepackage{booktabs}
\usepackage{comment}


\usepackage[hmargin=3cm,vmargin=6.0cm]{geometry}
%\topmargin=0cm
\topmargin=-2cm
\addtolength{\textheight}{6.5cm}
\addtolength{\textwidth}{2.0cm}
%\setlength{\leftmargin}{-5cm}
\setlength{\oddsidemargin}{0.0cm}
\setlength{\evensidemargin}{0.0cm}


\begin{document}

\section*{Student Information } 
%Write your full name and id number between the colon and newline
%Put one empty space character after colon and before newline
Full Name : Deniz Koluaçık \\
Id Number : 2310274 \\

% Write your answers below the section tags
\section*{Answer 1}
\begin{enumerate}[label=\textbf{\alph*)}]
	\item $ $
		\begin{enumerate}[label=(\roman*)]
			\item $D = A \cap (B \cup C)$
			\item $E = (A \cap B) \cup C$
			\item $F = (A \cap \overline{B})\cup(A\cap B\cap C)$

		\end{enumerate}
	\item $ $
	\begin{enumerate}[label=(\roman*)]
		\item Notice that the set $(A \times B) \times C$ consists of elements in the format $((a_i,b_i),ci)$ where $a_i \in A,\ b_i \in B,\ $and$\ c_i \in C$, meanwhile the set $A\times (B\times C)$ consists of elements in the format $(a_i, (b_i, c_i))$.\\

			Let us assume $A = \{1\},\ B=\{1\},\ $and $C=\{1\}$.\\
			$(A\times B)\times C = \{ ((1,1),1)\}$ and $A\times(B\times C)=\{(1,(1,1))\}$.\\
			Since $((1,1),1) \notin A\times(B \times C)$, we can say that $(A\times B)\times C \neq A\times(B\times C)$\\
	\item We can show that the two given expressions are equivalent by using a membership table.\\


		\begin{tabular}{*7{l}}
			\toprule
			$A$ & $B$ & $C$  & $A\cap B$ & $B\cap C$ & $(A\cap B)\cap C$ & $A\cap(B\cap C)$\\
			\midrule
			1 & 1 & 1 & 1 & 1 & 1 & 1\\
			1 & 1 & 0 & 1 & 0 & 0 & 0\\
			1 & 0 & 1 & 0 & 0 & 0 & 0\\
			1 & 0 & 0 & 0 & 0 & 0 & 0\\
			0 & 1 & 1 & 0 & 1 & 0 & 0\\
			0 & 1 & 0 & 0 & 0 & 0 & 0\\
			0 & 0 & 1 & 0 & 0 & 0 & 0\\
			0 & 0 & 0 & 0 & 0 & 0 & 0\\
			\bottomrule

		\end{tabular}\\

		We have shown that for all x the expressions on the last two columns have the same value.\newpage
	\item We will use the same approach here.\\

		\begin{tabular}{*7{l}}
			\toprule
			$A$ & $B$ & $C$  & $A\oplus B$ & $B\oplus C$ & $(A\oplus B)\oplus C$ & $A\oplus(B\oplus C)$\\
			\midrule
			1 & 1 & 1 & 0 & 0 & 1 & 1\\
			1 & 1 & 0 & 0 & 1 & 0 & 0\\
			1 & 0 & 1 & 1 & 1 & 0 & 0\\
			1 & 0 & 0 & 1 & 0 & 1 & 1\\
			0 & 1 & 1 & 1 & 0 & 0 & 0\\
			0 & 1 & 0 & 1 & 1 & 1 & 1\\
			0 & 0 & 1 & 0 & 1 & 1 & 1\\
			0 & 0 & 0 & 0 & 0 & 0 & 0\\
			\bottomrule
		\end{tabular}\\

		We have shown that for all x the expressions on the last two columns have the same value.
	\end{enumerate}
	
\end{enumerate}


\section*{Answer 2}
\begin{enumerate}[label=\textbf{\alph*)}]
	\item Let our domain of discourse be $A$.\\
		$f^{-1}(f(A_0)) = \{x| \exists y \in A_0(f(x) = f(y))$\\
		$\forall x(x\in A_0 \rightarrow \exists y \in A_0 (f(x) = f(y))$\\
		$\forall x(x\in A_0 \rightarrow x \in f^{-1}(f(A_0)) $\\
		Hence, $A_0 \subseteq f^{-1}(f(A_0))$.\\

		If $f$ is injective, or in other words, $\forall x \forall y(f(x) = f(y) \rightarrow x = y)$, we can say that
		\[\forall x(\exists y\in A_0 (f(x) = f(y) \rightarrow x = y \rightarrow x \in A_0))\]
		\[\forall x(x\in f^{-1}(f(A_0)) \rightarrow x \in A_0)\]
		\[f^{-1}(f(A_0)) \subseteq\ A_0\]
		Since $A_0 \subseteq f^{-1}(f(A_0))$ and $f^{-1}(f(A_0)) \subseteq A_0$, we can conclude by saying $A_0 = f^{-1}(f(A_0))$ if $f$ is injective.\\
	
	\item %Let our domain of discourse be $B$.\\
		$f(f^-1(B_0)) = \{f(a) | a\in A\ \exists b \in B_0 (f(a) = b)\}$\\
		Since all elements of $f(f^-1(B_0))$ are in $B_0$ ($f(a) = b \rightarrow f(a) \in B_0$), we can say that \[f(f^{-1}(B_0)) \subseteq B_0\]
		
		If $f$ is surjective, then it is not the case that there exists an $b \in B_0$ such that $f^{-1}(b) = \emptyset$, or in other words, there is always some $a \in A$ such that the function $f$ maps that $a$ to $b$ for any $b \in B_0$. Based on this, we can construct the claims,
		\[ \forall b\in B_0\ \exists a\in A(f^{-1}(b) = a \neq \emptyset)\]
		and finally,
		\[ \forall b\in B_0\ \exists a\in A(f(f^{-1}(b)) = b)\]
		Hence,
		\[B_0\subseteq f(f^{-1}(B_0)) \]
		Since $B_0\subseteq f(f^{-1}(B_0))$ and $f(f^{-1}(B_0)) \subseteq B_0$, we can finally say $B_0 = f(f^{-1}(B_0))$ if $F$ is surjective.\\
\end{enumerate}
		


\section*{Answer 3}
		If $A$ is countable, then we can define a new set $B \subseteq \mathbb{Z^+}$ such that $|B| = |A|$. We can define a function $g:B \rightarrow A$ so that $g$ is bijective since $|A| = |B|$.\\ 
		Based on this, we can define another function $f:\mathbb{Z^+} \rightarrow A$, 
		\begin{equation*}
			f(x) = \begin{cases}
				g(x) & \text{if $x \in B$}\\
				a & \text{if $x \notin B$}\\
			\end{cases}
		\end{equation*}
		where $a$ can be any element of the set $A$.\\
		The function $f$ is surjective, since $\forall a\in A\ \exists b\in B(f(b) = a)$ by the way we defined the function $g$.\\
		Therefore (i) $\rightarrow$ (ii).\\


		We can apply the same reasoning we used while defining the functions $f$ and $g$ in reverse. If there exists a function $f$ that is surjective. We can find a set $B \subseteq \mathbb{Z^+}$ such that all elemets of $B$ are mapped to a distinct element of $A$ and it is not the case that there exists an element of $A$ such that no element of $B$ is mapped to that element of $A$, or formally:
		\[ \forall b\in B\ \forall a\in B\ (f(a) = f(b) \rightarrow a = b) \land \forall a\in A\ \exists b\in B\ (f(b)=a)\]
		We can redefine $f$ with this restricted domain as $g: B \rightarrow A$. Notice that the function $g$ is no different than the one we previously defined, and $g$ is bijective, again. We can say since there exists a bijective function $g$, we can take the inverse of this bijective function $g^{-1}:A \rightarrow B$, which is also bijective. Finally by extending the codomain of our function $g^{-1}$ to $\mathbb{Z^+}$ we find a new function $f_1:A \rightarrow\mathbb{Z^+}$ that might be no longer surjective but is injective, nonetheless.\\
		Therefore, (ii) $\rightarrow$ (iii).\\
		
		For the last step, suppose that $A$ is uncountable. If it is uncountable then $|A| > |\mathbb{Z^+}|$ since $\mathbb{Z^+}$ is countable. But since there exists a injective function from $A$ to $\mathbb{Z^+}$, the cardinality of $A$ must be less than $\mathbb{Z^+}$ \textit{(see definition 2 on page 170 of the textbook)}. We reach a contradiction here. Therefore, $A$ is countable.
		Therefore, (iii) $\rightarrow$ (i).\\

		Finally, by showing (i) $\rightarrow$ (ii) $\rightarrow$ (iii) $\rightarrow$ (i), we have reached the conclusion (i), (ii), and (iii) are equivalent.
			
	
\begin{comment}




We show that (i) $\rightarrow$ (ii) $\rightarrow$ (iii) $\rightarrow$ (i).

$A$ is countable if and only if it has the same cardinality as $\mathbb{N}$, which is $\aleph_0$, and there is a one-to-one correspondance (or a bijection) between $A$ and $\mathbb{N}$.\\
We can say that there exists a function $f: \mathbb{N} \rightarrow A$ such that\\
$\forall x\in \mathbb{N}\ \forall y\in \mathbb{N} ((f(x) = f(y)) \rightarrow x=y) \land \forall z\in A\ \exists t\in \mathbb{N} f(t) = y$ if $A$ is countable.
Notice that 

If $A$ is countable, then we can find an enumeration such that each $n_i \in \mathbb{Z^+}$ corresponds to a $m_i \in A$. Therefore we have a bijective function $g:\mathbb{Z^+} \rightarrow A$. We can have a bijective function $g$ only if there exists a surjective function $f_1:\mathbb{Z^+} \rightarrow A$. Therefore, if (i) is true, then (ii) has to be true. ((i) $\rightarrow$ (ii)).\\ 
If there exists a $f_1 : \mathbb{Z^+} \rightarrow A$, 
\end{comment}


\section*{Answer 4}
\begin{enumerate}[label=\textbf{\alph*)}]
	\item If we can find a method of enumeration for finite binary strings, then we can say that the set of finite strings is countable.\\
	We can find a such enumeration, and that would be:\\
		\begin{enumerate}[label=\arabic*-]
		\item 0
		\item 1
		\item 00
		\item 01
		\item 10
		\item 11
		\item 000\\
		\vdots
	\end{enumerate}
		In this enumeration we enumerate each binary string according to their length and the integral value they represent (such as $(100)_2 = 4$) in an increasing order. An algorithm for finding the order of any finite string would be:
		\begin{equation*}
			\sum_{n=1}^{q(x)-1} 2^n + f(x)
		\end{equation*}
		Where $q(x)$ is the function that maps any finite binary string to its length and $f(x)$ is the function that maps any finite binary string to the integral value that it represents. For example, the order of the string $01010$ in this enumeration would be:\\
		\begin{equation*}
			\sum_{n=1}^{4} 2^n + 10 = 2 + 4 + 8 + 16 + 10 = 40
		\end{equation*}
		Hence, the set of finite binary strings is countable.
	\item We will use the \textbf{\textit{Cantor diagonalization argument}} to show that there is no enumeration for the set of infinite binary strings, and therefore, that it is uncountable.\\

	We begin with the assumption such enumeration exists. This enumeration would look like this:\\
	\begin{enumerate}[label=\arabic*-]
		\item $b_{11}b_{12}b_{13}b_{14}b_{15}b_{16}b_{16}\mathellipsis$
		\item $b_{21}b_{22}b_{23}b_{24}b_{25}b_{26}b_{26}\mathellipsis$
		\item $b_{31}b_{32}b_{33}b_{34}b_{35}b_{36}b_{36}\mathellipsis$
		\item $b_{41}b_{42}b_{43}b_{44}b_{45}b_{46}b_{46}\mathellipsis$\\
			\vdots

	\end{enumerate}
		Where $b_{nm}$ is the $m^{th}$ leftmost \textit{bit} of the $n^{th}$ string.\\
		We can construct an infinite binary string $d = d_1d_2d_3d_4d_5d_6d\mathellipsis$ following this rule.\\

		\[ d_i = \begin{cases}
			1 & \text{if $b_{ii} = 0$} \\
			0 & \text{if $b_{ii} = 1$} \\
		\end{cases}
		\]

		Which inherently does not appear anywhere in this enumeration. Since for any attempted enumeration we can construct a such $d$, we reach a contradiction with our initial assumption. Therefore, the set of the infinite binary strings is uncountable.
		


		
		

\end{enumerate}

\newpage
\section*{Answer 5}
\begin{enumerate}[label=\textbf{\alph*)}]
	\item $\log n! = \Theta(n\log n) \leftrightarrow (\log n! = 
		\Omega(n\log n)) \land (\log n! = O(n\log n))$
	\begin{itemize}
	\item $\log n! \stackrel{?}{=} O(n\log n)$\\

		We try to show that $\forall n > k_1 \log n! < c_1 \times n\log n$ for some $c_1$ and $k_1$.\\
		Let $c_1 = 1,\ k_1 = 10$.\\ 
		\begin{align*}
			\log n! &= \log (n \times (n-1) \times (n-2) \times \mathellipsis \times 2 \times 1) \\
			&= \log n + \log (n-1) + \log (n-2) + \mathellipsis + \log2 + \log1\\
			&< \underbrace{\log n + \mathellipsis + \log n}_{\text{n times}}=\log n^n= n\log n\qquad \forall n>10\\
		\end{align*}
			Therefore, $\log n! = O(n\log n).$\\
	\item $\log n! \stackrel{?}{=} \Omega(n\log n)$\\

		Now we will use a similar approach to find a pair of $c_2$ and $k_2$.\\

		\begin{align*}
			\log n! &= \log (n \times (n-1) \times (n-2) \times \mathellipsis \times 2 \times 1) \\
			&= \log n + \log (n-1) + \log (n-2) + \mathellipsis + \log2 + \log1\\
			&> \underbrace{\log\frac{n}{2} + \log\left(\frac{n}{2}+1\right) + \mathellipsis + \log(n-1) + \log n}_{\frac{n}{2}+1 \text{ times}}\\
			\log\frac{n}{2} + \log\left(\frac{n}{2}+1\right) + \mathellipsis + \log(n-1) + \log n &>  \underbrace{\log\frac{n}{2}+\mathellipsis+\log\frac{n}{2}}_{\frac{n}{2} \text{ times}} = \frac{n}{2}\log\frac{n}{2} = \frac{n}{2}(\log n - \log2)\\
			\frac{n}{2}(\log n - \log2) &= \Omega(n\log n)\qquad \text{ for $c_2 = 1/2$ and $k_2 = 10$.}\\
			\log n! &= \Omega(n\log n)
		\end{align*}

	\end{itemize}
		Finally, since $\log n! = \Omega(n\log n)$ and $\log n! = O(n\log n)$, we can conclude by saying $\log n! = \Theta(n\log n)$.\\
		\newpage
	\item Let $x_n$ and $y_n$ be the ratio between the $n^{th}$ and $(n+1)^{th}$ elements of the sequences $\{n!\}$ and $\{2^n\}$, respectively. By examining the ratio $\frac{x_n}{y_n}$ as $n$ approaches infinity, we can find which function grows faster.\\
		\[x_n = \frac{(n+1)!}{n!}\ ,\quad y_n = \frac{2^{n+1}}{2^n}\]

		\begin{equation*}
			\lim_{n \to \infty}{\frac{\frac{(n+1)!}{n!}}{\frac{2^{n+1}}{2^n}}}= \lim_{n \to \infty}{\frac{n+1}{2}}=\infty\\
		\end{equation*}
		What we have concluded from here is that the ratio between the $x_n$ and $y_n$ approaches to infinity as $n$ itself approaches to infinity. Hence, the sequence $\{n!\}$ increases faster than $\{2^n\}$, or in other words, the function $n!$ increases faster than the function $2^n$.
\end{enumerate}

\end{document}
